%%%%%%%%%%%%%%%%%%%%%%%%%%%%%%%%%%%%%%%%%%%%%%%%%%%%%%%%%%%%%%%%%%%%%%%%
%%%%%%%%%%%%%%%%%%%%%% Simple LaTeX CV Template %%%%%%%%%%%%%%%%%%%%%%%%
%%%%%%%%%%%%%%%%%%%%%%%%%%%%%%%%%%%%%%%%%%%%%%%%%%%%%%%%%%%%%%%%%%%%%%%%

%%%%%%%%%%%%%%%%%%%%%%%%%%%%%%%%%%%%%%%%%%%%%%%%%%%%%%%%%%%%%%%%%%%%%%%%
%% NOTE: If you find that it says                                     %%
%%                                                                    %%
%%                           1 of ??                                  %%
%%                                                                    %%
%% at the bottom of your first page, this means that the AUX file     %%
%% was not available when you ran LaTeX on this source. Simply RERUN  %%
%% LaTeX to get the ``??'' replaced with the number of the last page  %%
%% of the document. The AUX file will be generated on the first run   %%
%% of LaTeX and used on the second run to fill in all of the          %%
%% references.                                                        %%
%%%%%%%%%%%%%%%%%%%%%%%%%%%%%%%%%%%%%%%%%%%%%%%%%%%%%%%%%%%%%%%%%%%%%%%%

%%%%%%%%%%%%%%%%%%%%%%%%%%%% Document Setup %%%%%%%%%%%%%%%%%%%%%%%%%%%%

% Don't like 10pt? Try 11pt or 12pt
\documentclass[10pt]{article}


% This is a helpful package that puts math inside length specifications
\usepackage{calc}
\usepackage{comment}
\usepackage{multibbl}
\usepackage[nolist]{acronym}
\usepackage{graphicx}

% Layout: Puts the section titles on left side of page
\reversemarginpar

%
%         PAPER SIZE, PAGE NUMBER, AND DOCUMENT LAYOUT NOTES:
%
% The next \usepackage line changes the layout for CV style section
% headings as marginal notes. It also sets up the paper size as either
% letter or A4. By default, letter was used. If A4 paper is desired,
% comment out the letterpaper lines and uncomment the a4paper lines.
%
% As you can see, the margin widths and section title widths can be
% easily adjusted.
%
% ALSO: Notice that the includefoot option can be commented OUT in order
% to put the PAGE NUMBER *IN* the bottom margin. This will make the
% effective text area larger.
%
% IF YOU WISH TO REMOVE THE ``of LASTPAGE'' next to each page number,
% see the note about the +LP and -LP lines below. Comment out the +LP
% and uncomment the -LP.
%
% IF YOU WISH TO REMOVE PAGE NUMBERS, be sure that the includefoot line
% is uncommented and ALSO uncomment the \pagestyle{empty} a few lines
% below.
%

%% Use these lines for letter-sized paper
%\usepackage[paper=letterpaper,
%            %includefoot, % Uncomment to put page number above margin
%            marginparwidth=1.2in,     % Length of section titles
%            marginparsep=.05in,       % Space between titles and text
%            margin=1in,               % 1 inch margins
%            includemp]{geometry}

%% Use these lines for A4-sized paper
\usepackage[paper=a4paper,
            %includefoot, % Uncomment to put page number above margin
            marginparwidth=25.5mm,    % Length of section titles
            marginparsep=1.5mm,       % Space between titles and text
            margin=25mm,              % 25mm margins
            includemp]{geometry}

%% More layout: Get rid of indenting throughout entire document
\setlength{\parindent}{0in}

%% This gives us fun enumeration environments. compactitem will be nice.
\usepackage{paralist}

%% Reference the last page in the page number
%
% NOTE: comment the +LP line and uncomment the -LP line to have page
%       numbers without the ``of ##'' last page reference)
%
% NOTE: uncomment the \pagestyle{empty} line to get rid of all page
%       numbers (make sure includefoot is commented out above)
%
\usepackage{fancyhdr,lastpage}
\pagestyle{fancy}
%\pagestyle{empty}      % Uncomment this to get rid of page numbers
\fancyhf{}\renewcommand{\headrulewidth}{0pt}
\fancyfootoffset{\marginparsep+\marginparwidth}
\newlength{\footpageshift}
\setlength{\footpageshift}
          {0.5\textwidth+0.5\marginparsep+0.5\marginparwidth-2in}
\lfoot{\hspace{\footpageshift}%
       \parbox{4in}{\, \hfill %
                    \arabic{page} of \protect\pageref*{LastPage} % +LP
%                    \arabic{page}                               % -LP
                    \hfill \,}}

% Finally, give us PDF bookmarks
\usepackage{color,hyperref}
\definecolor{darkblue}{rgb}{0.0,0.0,0.3}
\hypersetup{colorlinks,breaklinks,
            linkcolor=darkblue,urlcolor=darkblue,
            anchorcolor=darkblue,citecolor=darkblue}
%\usepackage{bibentry}
%\nobibliography*
%%%%%%%%%%%%%%%%%%%%%%%% End Document Setup %%%%%%%%%%%%%%%%%%%%%%%%%%%%


%%%%%%%%%%%%%%%%%%%%%%%%%%% Helper Commands %%%%%%%%%%%%%%%%%%%%%%%%%%%%

% The title (name) with a horizontal rule under it
%
% Usage: \makeheading{name}
%
% Place at top of document. It should be the first thing.
\newcommand{\makeheading}[1]%
        {\hspace*{-\marginparsep minus \marginparwidth}%
         \begin{minipage}[t]{\textwidth+\marginparwidth+\marginparsep}%
                {\large \bfseries #1}\\[-0.15\baselineskip]%
                 \rule{\columnwidth}{1pt}%
         \end{minipage}}

% The section headings
%
% Usage: \section{section name}
%
% Follow this section IMMEDIATELY with the first line of the section
% text. Do not put whitespace in between. That is, do this:
%
%       \section{My Information}
%       Here is my information.
%
% and NOT this:
%
%       \section{My Information}
%
%       Here is my information.
%
% Otherwise the top of the section header will not line up with the top
% of the section. Of course, using a single comment character (%) on
% empty lines allows for the function of the first example with the
% readability of the second example.
\renewcommand{\section}[2]%
        {\pagebreak[2]\vspace{1.3\baselineskip}%
         \phantomsection\addcontentsline{toc}{section}{#1}%
         \hspace{0in}%
         \marginpar{
         \raggedright \scshape \textbf{#1}}#2}

% An itemize-style list with lots of space between items
\newenvironment{outerlist}[1][\enskip\textbullet]%
        {\begin{itemize}[#1]}{\end{itemize}%
         \vspace{-.6\baselineskip}}

% An environment IDENTICAL to outerlist that has better pre-list spacing
% when used as the first thing in a \section 
\newenvironment{lonelist}[1][\enskip\textbullet]%
        {\vspace{-\baselineskip}\begin{list}{#1}{%
        \setlength{\partopsep}{0pt}%
        \setlength{\topsep}{0pt}}}
        {\end{list}\vspace{-.6\baselineskip}}

% An itemize-style list with little space between items
\newenvironment{innerlist}[1][\enskip\textbullet]%
        {\begin{compactitem}[#1]}{\end{compactitem}}

% To add some paragraph space between lines.
% This also tells LaTeX to preferably break a page on one of these gaps
% if there is a needed pagebreak nearby.
\newcommand{\blankline}{\quad\pagebreak[2]}

%%%%%%%%%%%%%%%%%%%%%%%% End Helper Commands %%%%%%%%%%%%%%%%%%%%%%%%%%%



\makeatletter
\renewcommand\@biblabel[1]{\textbullet}
\makeatother









%%%%%%%%%%%%%%%%%%%%%%%%% Begin CV Document %%%%%%%%%%%%%%%%%%%%%%%%%%%%

\begin{document}

\newbibliography{pub}
\bibliographystyle{pub}{plainyr-rev}
\newbibliography{diss}
\bibliographystyle{diss}{plainyr-rev}
\newbibliography{tr}
\bibliographystyle{tr}{plainyr-rev}

\makeheading{Nikolaos Kyriazis}

%%%%%%%%%%%%%%%%%%%%%%%%%%%%%%%%%%%%%%%%%%%%%%%%%%%%%%%%%%%%%%%%
%      PERSONAL INFORMATION 
%%%%%%%%%%%%%%%%%%%%%%%%%%%%%%%%%%%%%%%%%%%%%%%%%%%%%%%%%%%%%%%%

\section{Personal Information}
\begin{minipage}[c]{0.8\textwidth}
\begin{innerlist}
\item Date of birth: December 11, 1982
\item Place of birth: Athens
\item Citizenship: Greek
\item Marital status: Married, two children
%\item For more information please refer to  \href{https://gr.linkedin.com/in/nikolaos-kyriazis-28abb0a1
%}{Google Scholar} or \href{https://gr.linkedin.com/in/nikolaos-kyriazis-28abb0a1
%}{LinkedIn} accounts
\end{innerlist}
\end{minipage}
\begin{minipage}[c]{0.2\textwidth}
\includegraphics[width=\textwidth]{avatar}
\end{minipage}

%%%%%%%%%%%%%%%%%%%%%%%%%%%%%%%%%%%%%%%%%%%%%%%%%%%%%%%%%%%%%%%%
%                                                                               CONTACT INFORMATION
%%%%%%%%%%%%%%%%%%%%%%%%%%%%%%%%%%%%%%%%%%%%%%%%%%%%%%%%%%%%%%%%

\section{Contact Information}
%
% NOTE: Mind where the & separators and \\ breaks are in the following
%       table.
%
% ALSO: \rcollength is the width of the right column of the table 
%       (adjust it to your liking; default is 1.85in).
%
\newlength{\rcollength}\setlength{\rcollength}{2.0in}%
%
\begin{tabular}[t]{@{}p{\textwidth-\rcollength}p{\rcollength}}
        
        \href{http://www.ics.forth.gr/cvrl}{\textbf{C}omputational \textbf{V}ision and \textbf{R}obotics \textbf{L}aboratory} & \\
        \href{http://www.ics.forth.gr}{\textbf{I}nstitute of \textbf{C}omputer \textbf{S}cience} & \\
        \href{http://www.forth.gr}{\textbf{FO}undation for \textbf{R}esearch and \textbf{T}echnology - \textbf{H}ellas}& \\
        \\
                                                                         Vasilika Vouton, P.O. Box 1385 & \textit{Voice:} +30 2811 392554 \\
Heraklion, Crete, Greece       & \textit{E-mail:} \href{mailto:kyriazis@ics.forth.gr}{nkyriazis@gmail.com}\\


\end{tabular}

%%%%%%%%%%%%%%%%%%%%%%%%%%%%%%%%%%%%%%%%%%%%%%%%%%%%%%%%%%%%%%%%
%                                                                               CINIZENSHIP
%%%%%%%%%%%%%%%%%%%%%%%%%%%%%%%%%%%%%%%%%%%%%%%%%%%%%%%%%%%%%%%%

%\section{Citizenship}
%
%Greek

%%%%%%%%%%%%%%%%%%%%%%%%%%%%%%%%%%%%%%%%%%%%%%%%%%%%%%%%%%%%%%%%
%                                                                               RESEARCH INTERESTS
%%%%%%%%%%%%%%%%%%%%%%%%%%%%%%%%%%%%%%%%%%%%%%%%%%%%%%%%%%%%%%%%

\section{Research Interests}
%
Computer vision, machine learning, software engineering, super-computing, computer graphics, optimization theory.

\section{Invited talks}

\begin{outerlist}
  \item[] ORamaVR  \hfill \textbf{17/7/2020}
  \begin{innerlist}
    \item Host: Associate Prof. George Papagiannakis (\ac{CSD}, \ac{UOC}), CEO/CTO/Co-founder
    \item Location: Switzerland (Virtual)
    \item Subject: Product-driven workflows, CI/CD
  \end{innerlist}
\end{outerlist}

\begin{outerlist}
  \item[] Max Planck Institute for Intelligent Systems (MPI)  \hfill \textbf{27/11/2014}
  \begin{innerlist}
    \item Host: Assistant Prof. Jeanette Bohg (Stanford)
    \item Location: T\"ubingen
    \item Subject: Computational methods for observing and understanding the interaction of human and robotic systems with objects of their environment
  \end{innerlist}
\end{outerlist}

%%%%%%%%%%%%%%%%%%%%%%%%%%%%%%%%%%%%%%%%%%%%%%%%%%%%%%%%%%%%%%%%
%                                                                               ACADEMIC STATUS
%%%%%%%%%%%%%%%%%%%%%%%%%%%%%%%%%%%%%%%%%%%%%%%%%%%%%%%%%%%%%%%%

%\section{Academic status}
%
%\href{http://www.csd.uoc.gr/}{Computer Science Department}\\
%\href{http://www.uoc.gr/}{University of Crete}, 
%Heraklion, Crete, Greece

%\begin{outerlist}
%        \item[] PhD student \hfill \textbf{February 2009 - June 2014}
%        \item[] Graduate student \hfill \textbf{February 2006 - December 2008}
%        \item[] Undergraduate student  \hfill \textbf{September 2001 - February 2006}
%\end{outerlist}

\section{Awards}
\begin{outerlist}
\item[] \textbf{Research award from Facebook Reality Labs} \hfill \textbf{23/1/2020}
    \begin{innerlist}
        \item Awarded on the basis of the impact of work on human-computer interactions
        \item Award: Research gift (grant)
    \end{innerlist}
\item[] \textbf{NVidia hardware grant} \hfill \textbf{16/1/2019}
    \begin{innerlist}
        \item Call: NVidia GPU\ grant
        \item Proposal: High-fidelity simulation of image formation and pushing the envelope in Deep-Learning-based
3d hand estimation through analysis by synthesis
        \item Award: NVidia Titan V GPU
    \end{innerlist}
\item[] \textbf{NVidia hardware grant} \hfill \textbf{9/8/2014}
    \begin{innerlist}
        \item Call: Tegra\textregistered~K1 CUDA Hardware Donation Program for Researchers
        \item Proposal: CUDA-enabled model-based framework on embedded systems        \item Award: NVidia Jetson TK1 DevKit
    \end{innerlist}
\item[] \textbf{Ph.D. with Honors} \hfill \textbf{14/6/2014}
    \begin{innerlist}
        \item Awarded Ph.D. with honors, by unanimous vote of the examination committee.
%        \item Scope: Final examination for the Ph.D.
    \end{innerlist}

\item[] \textbf{Young Researcher Award 2012--2013} \hfill \textbf{15/7/2013}
    \begin{innerlist}
%        \item Awarded and sponsored by the University of Crete
%        \item Scope: Graduate students (MsC, PhD) ss all departments of the 5 schools which comprise the University of Crete
%        \item Object: To award students whose research has significantly contributed to the advancement of their research field
        \item Sponsored by the University of Crete. Awarded to the top performing graduate student (MSc, PhD) ss all departments of the 5 schools which comprise the University of Crete.
    \end{innerlist}
    
    \item[] \textbf{1st Prize} in \textbf{\href{http://sites.google.com/a/chalearn.org/gesturechallenge}{ChaLearn Gesture Challenge 2012}} \hfill \textbf{11/11/2012}
    \begin{innerlist}
        \item Sponsored by Microsoft, Redmond, USA. Awarded in the Demo\ Competition of the Qualitative Evaluation track, for the best gesture recognition demo. Hosted in Tsukuba, Japan in conjunction with ICPR 2012 (Gesture Recognition Workshop).
%        \item Hosted in Tsukuba, Japan in conjunction with ICPR 2012 (Gesture Recognition Workshop)
%        \item Participation: Contestant in the Qualitative Evaluation (Demo Competition)
%        \item Presentation: \href{http://cvrlcode.ics.forth.gr/handtracking}{``Giving a Hand to Kinect''}
%        \item Sponsored by Microsoft, Redmond, USA
    \end{innerlist}
\end{outerlist}

%%%%%%%%%%%%%%%%%%%%%%%%%%%%%%%%%%%%%%%%%%%%%%%%%%%%%%%%%%%%%%%%
%                                                                               EDUCATION
%%%%%%%%%%%%%%%%%%%%%%%%%%%%%%%%%%%%%%%%%%%%%%%%%%%%%%%%%%%%%%%%

\section{Education}
%
\href{http://www.uoc.gr/}{\textbf{University of Crete}}, 
Heraklion, Crete, Greece

\begin{outerlist}

\item[] Ph.D., 
        \href{http://www.csd.uoc.gr/}
             {\ac{CSD}, \ac{UOC}} 
        \hfill \textbf{14/6/2014}
        \begin{innerlist}
        \item Thesis: Computational methods for observing and understanding the interaction of human and robotic systems with objects of their environment        \item Supervisor: 
              \href{http://www.ics.forth.gr/~argyros}
                   {Professor Antonis Argyros}
        \item Area of Study: Computer Vision
        \end{innerlist}

\item[] M.Sc., 
        \href{http://www.csd.uoc.gr/}
             {\ac{CSD}, \ac{UOC}} 
        \hfill \textbf{1/4/2009}
        \begin{innerlist}
        \item Thesis: Context Free Grammar Induction via Observation of Structured Time Processes
        \item Supervisor: 
              \href{http://www.ics.forth.gr/~argyros}
                   {Associate Professor Antonis Argyros}
        \item Area of Study: Machine Learning, Computer Vision
        \end{innerlist}

\item[] B.Sc., 
        \href{http://www.csd.uoc.gr/}
             {\ac{CSD}, \ac{UOC}}, \hfill \textbf{1/2/2006}
        \begin{innerlist}
        \item Specialization in Software Engineering and Information Systems
        \item Minor in Information Systems
        \end{innerlist}

\end{outerlist}


%%%%%%%%%%%%%%%%%%%%%%%%%%%%%%%%%%%%%%%%%%%%%%%%%%%%%%%%%%%%%%%%
%                                        ATTENDED COURSES
%%%%%%%%%%%%%%%%%%%%%%%%%%%%%%%%%%%%%%%%%%%%%%%%%%%%%%%%%%%%%%%%

\begin{comment}
\section{Relevant Attended Courses}
%
\href{http://www.csd.uoc.gr}{\textbf{C}omputer \textbf{S}cience \textbf{D}epartment}\\
School of Sciences and Technology\\
\href{http://www.uoc.gr}{\textbf{U}niversity \textbf{O}f \textbf{C}rete}

\begin{outerlist}
   \item[] Course \href{http://www.csd.uoc.gr/~hy672}{CS672} \hfill \textbf{Spring 2007}
   \begin{innerlist}
       \item Course title: Advanced Topics of Computer Vision
       \item Teacher: Associate Professor \href{mailto:argyros@csd.uoc.gr}{A. Argyros}
       \item Contents
       \begin{innerlist}
           \item Image Acquisition and Optical Sensors
           \item Low Level Algorithms for Vision
           \item Image Features
           \item Image Segmentation and Clustering
           \item Image Color and Texture
           \item Motion Detection and Object Tracking
           \item Multi-View Geometry
           \item 3D reconstruction
           \item Shape Representation and Object Description
           \item Optical Flow
           \item Object Recognition
           \item Content-based Image Retrieval
           \item Statistical Models and Machine Learning for Computer Vision
           \item Active and Robotic Vision
           \item Face and Gesture Recognition
           \item Cognitive and Biologically Inspired Vision
       \end{innerlist}
   \end{innerlist}

   \item[] Course \href{http://www.csd.uoc.gr/~hy577}{CS577} \hfill \textbf{Winter 2006}
   \begin{innerlist}
       \item Course title: Machine Learning
       \item Teacher: Assistant Professor \href{mailto:tsamard@csd.uoc.gr}{I. Tsamardinos}
       \item Contents
       \begin{innerlist}
           \item Statistical Machine Learning
           \item Supervised Learning
           \item Classification and Regression -- Learning Algorithms (Linear/Non Linear Regression,
               Neural Networks, Support Vector Machines, more)
           \item Performance and Accuracy Estimation (Accuracy, ROC, cross-validation, VC-dimension, more)
           \item Machine Learning in Practice (Data acquisition, sorting and representation, more)
           \item Unsupervised Learning (Clustering, Rule Learning)
           \item More Issues (Reinforcement Learning, Markov-Chains, Active Learning, more)
       \end{innerlist}
   \end{innerlist}
\end{outerlist}
\end{comment}

%%%%%%%%%%%%%%%%%%%%%%%%%%%%%%%%%%%%%%%%%%%%%%%%%%%%%%%%%%%%%%%%
%                                                                               RESEARCH AND DEVELOPMENT EXPERIENCE
%%%%%%%%%%%%%%%%%%%%%%%%%%%%%%%%%%%%%%%%%%%%%%%%%%%%%%%%%%%%%%%%

\section{R\&D Experience}
%
\begin{outerlist}
        \item[] \textbf{Senior Computer Vision Engineer} at \ac{CVRL}, \ac{ICS}, \ac{FORTH} \hfill \textbf{1/2/2020 - Present}
        \begin{innerlist}
                \item Conducted and published research in the field of Computer Vision, regarding 3D tracking of hands and objects in interaction. Research was conducted in and supported by EU-funded projects, under the supervision and with the collaboration of Professor A. Argyros.
        \end{innerlist}

        \item[] \textbf{Computer Vision Engineer} \\at \textbf{Oculus VR--Facebook}   \hfill \textbf{16/4/2017 - 17/1/2019}
        \begin{innerlist}
        \item Pushed the envelope in real-time robust 3d hand tracking, in the context of Virtual Reality (VR), combining deep-learning and model-based methods.
        \end{innerlist}

        \item[] \textbf{Postdoctoral researcher} at \ac{CVRL}, \ac{ICS}, \ac{FORTH} \hfill \textbf{15/6/2014 - 21/12/2016}
        \begin{innerlist}
                \item Conducted and published research in the field of Computer Vision, regarding 3D tracking of hands and objects in interaction. Research was conducted in and supported by EU-funded projects, under the supervision and with the collaboration of Professor A. Argyros.
%                \item Laboratory head: Professor \href{mailto:trahania@ics.forth.gr}{P. Trahanias}
%                \item Under the supervision of Professor \href{mailto:argyros@ics.forth.gr}{A. Argyros}
%                \item Research areas: Computer vision, machine learning, computer graphics, super-computing
        \end{innerlist}
\item[] \textbf{Internship} at \ac{DRZ} \hfill \textbf{1/10/2013 - 21/12/2013}
        \begin{innerlist}
                \item Contributed in acquisition and representation of real world data, towards the development of algorithms that leverage this data to generate stunning visual effects, under the supervision of \href{mailto:thabo.beeler @disneyresearch.com}{Dr. Thabo Beeler}, in the \href{http://www.disneyresearch.com/research-labs/disney-research-zurich/dr-zurich-capture-effects/}{Capture \& Effects} laboratory. During this period I had the opportunity to work with highly talented people and obtain experience in new tools and methodologies
regarding high fidelity performance capture.        \end{innerlist}

        \item[] \textbf{Scholarship} from \ac{CVRL}, \ac{ICS}, \ac{FORTH} \hfill \textbf{1/9/2006 - 14/6/2014}
        \begin{innerlist}
                \item Postgraduate scholarships covering M.Sc. and Ph.D. studies. Conducted and published research in the fields of Computer Vision and Machine Learning, regarding 3D tracking of hands and objects in interaction and grammar induction. Research was conducted in and supported by EU-funded projects, under the supervision and with the collaboration of Professor A. Argyros.
During this period I obtained a significant amount of experience regarding several aspects of research \& development, such as working and collaborating within a team, conducting research methodically, reporting research and findings, developing and integrating software and attracting/acquiring research funding through research proposals.
%                \item Laboratory head: Professor \href{mailto:trahania@ics.forth.gr}{P. Trahanias}
%                \item Under the supervision of Associate Professor \href{mailto:argyros@ics.forth.gr}{A. Argyros}
%                \item Research areas: Computer vision, machine learning, computer graphics, super-computing
        \end{innerlist}

        \item[] {\bf \href{http://rapid-project.eu/}{RAPID: }}\textit{Heterogeneous Secure Multi-level Remote Acceleration Service for Low-Power Integrated Systems and Devices} \\ (H2020-ICT-644312) \hfill \textbf{1/1/2015 - 31/12/2016}
        \begin{innerlist}
            \item Assistance in proposal preparation
            \item Work on providing a client-server distributed decomposition of the standalone 3D hand tracking application developed by \ac{CVRL}. Two implementations were provided, for comparison, a custom one based on basic \ac{RPC} and one based on the proposed RAPID infrastructure.
%            \item Associated Lab: \href{http://www.ics.forth.gr/cvrl}{\textbf{C}omputational \textbf{V}ision and \textbf{R}obotics \textbf{L}aboratory}, \href{http://www.ics.forth.gr/}{ICS}, \href{http://www.forth.gr/}{FORTH}
        \end{innerlist}
                        
        \item[] {\bf \href{http://www.wearhap.eu/}{WEARHAP: }}\textit{Wearable Haptics for Humans and Robots} \\ (FP7-ICT-2011-9) \hfill \textbf{1/3/2013 - 31/12/2016}
        \begin{innerlist}
            \item Assistance in proposal preparation
            \item Work on hand pose estimation based on dimensionality reduction
%            \item Associated Lab: \href{http://www.ics.forth.gr/cvrl}{\textbf{C}omputational \textbf{V}ision and \textbf{R}obotics \textbf{L}aboratory}, \href{http://www.ics.forth.gr/}{ICS}, \href{http://www.forth.gr/}{FORTH}
        \end{innerlist}
        
        \item[] {\bf \href{http://robohow.eu/}{RoboHow.cog: }}\textit{Web-enabled and Experience-based Cognitive Robots that Learn
Complex Everyday Manipulation Tasks } \hfill \textbf{1/2/2012 - 31/5/2016}\\ (FP7-288533)
        \begin{innerlist}
            \item Assistance in proposal preparation
            \item Observation of human demonstrations, extraction and symbolic representation of information
%            \item Associated Lab: \href{http://www.ics.forth.gr/cvrl}{\textbf{C}omputational \textbf{V}ision and \textbf{R}obotics \textbf{L}aboratory}, \href{http://www.ics.forth.gr/}{ICS}, \href{http://www.forth.gr/}{FORTH}
        \end{innerlist}
        
                \item[] \textbf{\href{http://www.grasp-project.eu/}{GRASP}}:\textit{Emergence of Cognitive Grasping through Introspection, Emulation and Surprise (IST-FP7-IP-215821)} \hfill \textbf{1/9/2007 - 29/2/2012}
        \begin{innerlist}
                \item 3D hand tracking from various visual sensing modalities
%                \item Associated Lab: \href{http://www.ics.forth.gr/cvrl}{\textbf{C}omputational \textbf{V}ision and \textbf{R}obotics \textbf{L}aboratory}, \href{http://www.ics.forth.gr/}{ICS}, \href{http://www.forth.gr/}{FORTH}
        \end{innerlist}
        
%        \item[] \textbf{\href{http://www.vector-project.com/}{VECTOR}}:\textit{\textbf{V}ersatile \textbf{E}ndoscopic \textbf{C}apsule for gastrointestinal \textbf{T}um\textbf{O}r \textbf{R}ecognition and therapy (IST-FP6-IP-033970)} \hfill \textbf{Sept. 2007 - Feb. 2008}
%        \begin{innerlist}
%                \item Real-time simulation of perspective and panoramic camera models for  intestinal image acquisition and processing study
%                \item Associated Lab: \href{http://www.ics.forth.gr/cvrl}{\textbf{C}omputational \textbf{V}ision and \textbf{R}obotics \textbf{L}aboratory}, \href{http://www.ics.forth.gr/}{ICS}, \href{http://www.forth.gr/}{FORTH}
%        \end{innerlist}
        
        \item[] \textbf{Scholarship} from \hfill \textbf{1/9/2005 - 30/9/2006}\\
\ac{HCI}, \ac{ICS}, \ac{FORTH} 
        \begin{innerlist}
                \item Undergraduate and postgraduate scholarships. Studied and applied Software Engineering in the context of developing and shipping universally accessible software, under the supervision of Associate Professor \href{mailto:as@ics.forth.gr}{A. Savidis} and Professor \href{mailto:cs@ics.forth.gr}{C. Stephanidis}. During this period of time I\ acquainted myself with problems and solutions behind product-oriented software engineering.
%                \item Laboratory head: Professor \href{mailto:cs@ics.forth.gr}{C. Stephanidis}
%                \item Under the supervision of Associate Professor \href{mailto:as@ics.forth.gr}{A. Savidis}
%                \item Research areas: Software engineering, accessible software
        \end{innerlist}

                \item[] \textbf{STARLIGHT}:\textit{Design and development of a}\\
                                            \textit{commercial platform for interactive electronic} \hfill \textbf{\textbf{1/2/2006 - 28/2/2010}}\\
                                            \textit{books for sighted, low-vision and blind users}
        \begin{innerlist}
                \item Development of the book editor application, which is used to generate the content for the interactive electronic books.
%                \item Associated Lab: \href{http://www.ics.forth.gr/hci}{\textbf{H}uman \textbf{C}omputer \textbf{I}nteraction Laboratory}, \href{http://www.ics.forth.gr/}{ICS}, \href{http://www.forth.gr/}{FORTH}
        \end{innerlist}
        
\end{outerlist}

\section{Impact}
\begin{outerlist}
\item[] \href{https://scholar.google.gr/citations?user=fUsz0D8AAAAJ}{Google Scholar} (last update: 3/8/2020)
\begin{innerlist}
\item Citations: 2050, h-index: 13
\item Most cited visual hand tracking paper of all time: ``Efficient model-based 3D tracking of hand articulations using Kinect.'' (1045 citations)
\end{innerlist}
\end{outerlist}


%%%%%%%%%%%%%%%%%%%%%%%%%%%%%%%%%%%%%%%%%%%%%%%%%%%%%%%%%%%%%%%%
%                                                                               TEACHING EXPERIENCE
%%%%%%%%%%%%%%%%%%%%%%%%%%%%%%%%%%%%%%%%%%%%%%%%%%%%%%%%%%%%%%%%

\begin{comment}
\section{Teaching Experience}
%
\href{http://www.csd.uoc.gr}{\textbf{C}omputer \textbf{S}cience \textbf{D}epartment}\\
School of Sciences and Technology\\
\href{http://www.uoc.gr}{\textbf{U}niversity \textbf{O}f \textbf{C}rete}

\begin{outerlist}
        \item[] Teaching assistant in \href{http://www.csd.uoc.gr/~hy100}{CS100} \hfill \textbf{Spring 2006}, \textbf{Winter 2011}
        \begin{innerlist}
            \item In the course titled ``Programming with the C language'' I\ was responsible for recitation, preparing programming exercises and grading.
        \end{innerlist}
        
        \item[] Teaching assistant in \href{http://www.ics.forth.gr/~argyros/cs472.html}{CS472} \hfill \textbf{Spring 2008}, \textbf{Spring 2009},

 \hfill \textbf{Spring 2010, Spring 2011, Spring 2012, Spring 2013}        
        \begin{innerlist}
                \item In the course titled ``Computer Vision'' I was responsible for recitation, exercise preparation and students' project guidance.
        \end{innerlist}
        
        \item[] Teaching assistant in \href{http://elearn.uoc.gr/course/view.php?id=137}{CS387} \hfill \textbf{Winter 2009}
        \begin{innerlist}
                \item In the course titled ``Introduction to Artificial Intelligence'' I\ was responsible for recitation and exercise grading.
        \end{innerlist}

                \item[] Teaching assistant in \href{http://www.ics.forth.gr/~argyros/cs672.html}{CS672} \hfill \textbf{Winter 2008}
        \begin{innerlist}
                \item In the course titled ``Advanced Topics on Computer Vision '' I was responsible for recitation, exercise preparation, and students' project guidance.
        \end{innerlist}
        
                \item[] Teaching assistant in \href{http://www.csd.uoc.gr/~hy358}{CS358} \hfill \textbf{Spring 2007, Winter 2011}
        \begin{innerlist}
                \item In the course titled ``Computer Graphics'' I was responsible for recitation, exercise administration, preparation of programming exercises, grading, and students' project guidance.
        \end{innerlist}
        
                \item[] Teaching assistant in \href{http://www.csd.uoc.gr/~hy475}{CS475} \hfill \textbf{Winter 2006}
        \begin{innerlist}
                \item In the course titled ``Autonomous Robot Navigation'' I was responsible for the definition of the students' project and the students' project guidance.
        \end{innerlist}
        
        \item[] Teaching assistant in \href{http://www.csd.uoc.gr/~hy471}{CS471} \hfill \textbf{Winter 2013}
        \begin{innerlist}
                \item In the course titled ``Digital Image Processing'' I was responsible for recitation, preparing programming exercises and grading.
        \end{innerlist}
        
\end{outerlist}
\end{comment}

%%%%%%%%%%%%%%%%%%%%%%%%%%%%%%%%%%%%%%%%%%%%%%%%%%%%%%%%%%%%%%%%
%       ORGANIZATION OF CONFERENCES, SYMPOSIA, WORKSHOPS AND TRAINING
%%%%%%%%%%%%%%%%%%%%%%%%%%%%%%%%%%%%%%%%%%%%%%%%%%%%%%%%%%%%%%%%

\section{Organization of Scientific Events}
\begin{outerlist}
        \item[] ECCV 2010, 11th European Conference on Computer Vision
        \begin{innerlist}
            \item Hosted in Hersonissos, Crete, Greece \hfill \textbf{5/9/2010 - 11/9/2010}
            \item Participation: Coordination and administration assistance, attendant
        \end{innerlist}

        \item[] Eurographics 2008, the 29th annual conference of the European Association for Computer Graphics
                \begin{innerlist}
                        \item Hosted in Heraklion, Crete \hfill  \textbf{14/4/2008 - 18/4/2008}
                        \item Participation: Coordination and administration assistance, attendant
                \end{innerlist}

\end{outerlist}

%%%%%%%%%%%%%%%%%%%%%%%%%%%%%%%%%%%%%%%%%%%%%%%%%%%%%%%%%%%%%%%%
%       PARTICIPATION IN CONFERENCES, SYMPOSIA, WORKSHOPS AND TRAINING
%%%%%%%%%%%%%%%%%%%%%%%%%%%%%%%%%%%%%%%%%%%%%%%%%%%%%%%%%%%%%%%%

\section{Participation in Scientific Events}
%
\begin{outerlist}
        
        \item[] ICVSS 09, International Computer Vision Summer School
                \begin{innerlist}
                        \item Hosted in Sicily, Italy \hfill \textbf{6/7/2009 - 11/7/2009}
                        \item Participation: attendant
                        \item Renumeration : 2nd best grade in the final examinations
                \end{innerlist}

                
        \item[] The \textbf{Onassis Foundation} science lecture series:
                \textit{Robots Intelligently Interacting With People}
                \begin{innerlist}
                        \item Hosted in Heraklion, Crete \hfill \textbf{24/7/2006 - 28/7/2006}
                        \item Participation: Attendant
                \end{innerlist}
\end{outerlist}

%%%%%%%%%%%%%%%%%%%%%%%%%%%%%%%%%%%%%%%%%%%%%%%%%%%%%%%%%%%%%%%%
%       REVIEWING FOR SCIENTIFIC JOURNALS
%%%%%%%%%%%%%%%%%%%%%%%%%%%%%%%%%%%%%%%%%%%%%%%%%%%%%%%%%%%%%%%%

\section{Reviewing for Scientific Journals}
%
\begin{outerlist}
        \item []
    \begin{innerlist}
        \item \ac{CVIU} -- Elsevier
        \item \ac{CAD} - Elsevier
        \item IEEE \ac{CYB}
        \item IEEE \ac{THMS}
        \item IEEE \ac{TPAMI}
        \item \ac{IJCV} -- Springer

    \end{innerlist}
\end{outerlist}

%%%%%%%%%%%%%%%%%%%%%%%%%%%%%%%%%%%%%%%%%%%%%%%%%%%%%%%%%%%%%%%%
%       REVIEWING FOR CONFERENCES AND WORKSHOPS
%%%%%%%%%%%%%%%%%%%%%%%%%%%%%%%%%%%%%%%%%%%%%%%%%%%%%%%%%%%%%%%%

\section{Reviewing for Conferences and Workshops}
%
\begin{outerlist}
        \item []
    \begin{innerlist}
        \item \ac{ECCV}
        \item IEEE/CVF \ac{CVPR}
        \item IEEE/CVF \ac{ICCV}
        \item IEEE \ac{ICRA}
        \item IEEE Computer Society Workshop on Observing and understanding hands in action (HANDS 2015 in conjunction with CVPR 2015, HANDS 2016 in conjunction with CVPR 2016)
    \end{innerlist}
\end{outerlist}

%%%%%%%%%%%%%%%%%%%%%%%%%%%%%%%%%%%%%%%%%%%%%%%%%%%%%%%%%%%%%%%%
%                                                                                       LANGUAGES
%%%%%%%%%%%%%%%%%%%%%%%%%%%%%%%%%%%%%%%%%%%%%%%%%%%%%%%%%%%%%%%%

\section{Languages}
%
\begin{outerlist}
        \item[] Greek : excellent (native tongue)
        \item[] English : fluent (Certificate of Proficiency awarded by the University of Michigan)
        \item[] French : good (D.E.L.F.)
\end{outerlist}


%%%%%%%%%%%%%%%%%%%%%%%%%%%%%%%%%%%%%%%%%%%%%%%%%%%%%%%%%%%%%%%%
%                                                                               TECHNICAL SKILLS
%%%%%%%%%%%%%%%%%%%%%%%%%%%%%%%%%%%%%%%%%%%%%%%%%%%%%%%%%%%%%%%%

\begin{comment}
\section{Technical Skills} 
%
\begin{outerlist}
        \item[] Programming aspects
        \begin{innerlist}
                \item Object-oriented programming
                \item Functional programming
                \item Generic programming
        \end{innerlist}
        
        \item[] Programming languages
        \begin{innerlist}
                \item Compiled languages: C, C++
                \item Interpreted languages: Java, Python, Javascript
                \item Special purpose languages: MATLAB,  Mathematica,  CUDA
                \item Tools: Language theory, compiler compilers
        \end{innerlist}
        
        \item[] Development environments
        \begin{innerlist}
                \item Windows: Visual Studio series, Eclipse
                \item Unix: gcc, g++, CMake, Makefile, Eclipse, KDdevelop
                \item GPU: nvcc
        \end{innerlist}
        
        \item[] Libraries
        \begin{innerlist}
                \item \href{http://www.sgi.com/tech/stl/}{Standard Template Library (STL)} (C++), \href{http://www.boost.org/}{boost} (C++), \href{http://www.wxwidgets.org/}{wxWidgets} (C++), \href{http://www.microsoft.com/}{DirectX} (C++), \href{http://www.opengl.org/}{OpenGL} (C++), \href{http://www.python.org/}{Python (integration / embedding)} (C++), \href{http://vxl.sourceforge.net/}{VXL} (C++), \href{http://java.sun.com/}{Java API (Java)}, \href{http://sourceforge.net/projects/opencvlibrary}{OPENCV} (C++, Python), \href{http://code.google.com/p/thrust/}{Thrust} (C++,\ CUDA), \href{http://code.google.com/p/cudpp/}{CUDPP} (C++,\ CUDA), \href{http://www.numpy.org/}{NumPy} (Python), \href{http://www.scipy.org/}{SciPy} (Python), \href{http://matplotlib.org/}{matplotlib} (Python), \href{http://scikit-learn.org/stable/}{scikit-learn} (Python), \href{https://code.google.com/p/ceres-solver/}{ceres-solver} (C++)
        \end{innerlist}
        
        \item[] Applications
        \begin{innerlist}
                \item Microsoft Office Suite series
                \item Latex
                \item MATLAB
                \item Mathematica
        \end{innerlist}
\end{outerlist}
\end{comment}

%%%%%%%%%%%%%%%%%%%%%%%%%%%%%%%%%%%%%%%%%%%%%%%%%%%%%%%%%%%%%%%%
%                                                                               MATHEMATICAL EXPERTISE
%%%%%%%%%%%%%%%%%%%%%%%%%%%%%%%%%%%%%%%%%%%%%%%%%%%%%%%%%%%%%%%%

\begin{comment}
\section{Mathematical Background}
%
Algebraic analysis, discrete mathematics, linear algebra, optimization, \\ machine learning.
\end{comment}

\section{Publications}
\nocite{pub}{*}
%\renewcommand\refname{\vskip -1cm}
%\nocite{*}

\begingroup
\renewcommand{\section}[2]{}%
\bibliography{pub}{publications}{References to books}
\endgroup
%\bibliographystyle{plain}
%\bibliography{publications}

%%%%%%%%%%%%%%%%%%%%%%%%%%%%%%%%%%%%%%%%%%%%%%%%%%%%%%%%%%%%%%%%

\section{Dissertations}
\nocite{diss}{*}

\begingroup
\renewcommand{\section}[2]{}%
\bibliography{diss}{dissertations}{Dissertations}
\endgroup

%%%%%%%%%%%%%%%%%%%%%%%%%%%%%%%%%%%%%%%%%%%%%%%%%%%%%%%%%%%%%%%%

\section{Technical reports}
\nocite{tr}{*}

\begingroup
\renewcommand{\section}[2]{}%
\bibliography{tr}{treports}{Technical reports}
\endgroup


%%%%%%%%%%%%%%%%%%%%%%%%%%%%%%%%%%%%%%%%%%%%%%%%%%%%%%%%%%%%%%%%
%                                                                               MOTIVATION
%%%%%%%%%%%%%%%%%%%%%%%%%%%%%%%%%%%%%%%%%%%%%%%%%%%%%%%%%%%%%%%%

\begin{comment}
\section{Motivation}
%
%The most recent trend in Computer Vision (CV) is the employment of Machine Learning (ML) theory and tools. ML is incorporated so as to enhance the autonomy (auto-adjustment) of CV systems as well as their performance. ML itself is defined to be the ability of performance amelioration of a machine in an environment, that comes through observation and experiencing in general. Thus, in a manner, CV and ML can be thought of as complementary and co-beneficial. Therefore, the wide range of CV problems, that have gained the attention of a great part of the academia, has also lead to the generation or adaptation of a variety of ML tools, that target specifically these problems. From my point of view, the definition of a rich ML tool arsenal is essential when views of certain problems need to be examined or when the appropriateness of the tools incorporated needs to be ascertained.\\
%
%As a young researcher I have witnessed that, in most cases, the ability to solve problems well depends on the capableness to pose the right question, i.e. define the most appropriate objective function, and of course on the competence to optimize it. Both aspects raise a series of issues, upon which there is a need for spherical apprehension. Thus, when great problems, such as scene understanding, scene and knowledge representation and active learning, come into discussion there also comes a need for appropriate tools, such as learning methods, representations, relational models, reasoning methods and so on.\\


The interest that drives me into participating in  \href{http://svg.dmi.unict.it/icvss2010/}{International Computer Vision Summer School} derives from my occupation with problems that are relevant to the published syllabus. My MSc thesis focuses on ML and targets the generic application of learning and incremental online representation building of knowledge for timely structured observable systems. I am very interested in the full variety of the matters in the syllabus and the reason I apply for this summer school, and hope to be accepted, is my belief that it will be ultimately beneficial for me. I am at the very start of my PhD and I look forward to be informed on the methods and tools listed, since the area I will be researching in is very strongly correlated to the most part of the lectures' and tutorials' content. To be more specific, during my immediate-future academic activities I will be researching scene understanding and interactive learning, by demonstration or experimentation.

This year's ICVSS focuses on fundamental Computer Vision (CV) issues. I am very interested in participating in a summer school that would offer me both a broad view of the topics under consideration and the opportunity to attend lectures being carried out by the very innovators in the respective fields. I also look forward to the opportunity of direct discussion, not only with the distinguished speakers but also with other people (students, attendants etc.) that probably share the same research interests. This will hopefully provide me with better understanding, originating from different perspectives. Having attended ICVSS09 is also motivating, since I have experienced, first-hand, the high standards of this summer school series. Last ICVSS has been mostly beneficial for me and this is an important drive for applying for this year's summer school as well.


\end{comment}

\begin{acronym}
  \acro{CVRL}{\href{http://www.ics.forth.gr/cvrl/}{Computer Vision and Robotics Laboratory}}
  \acro{ICS}{\href{http://www.ics.forth.gr/}{Institute of Computer Science}}
  \acro{FORTH}{\href{http://www.forth.gr/}{Foundation for Research and Technology -- Hellas}}
  \acro{DRZ}{\href{http://www.disneyresearch.com/research-labs/disney-research-zurich/} {Disney Research Zurich}}
  \acro{HCI}{\href{http://www.ics.forth.gr/hci/}{Human Computer Interface}}
  \acro{RPC}{Remote Procedure Call}
  \acro{UOC}{University of Crete}
  \acro{CSD}{Computer Science Department}
  \acro{ECCV}{European Conference on Computer Vision}
  \acro{CVPR}{Computer Vision and Pattern Recognition}
  \acro{ICCV}{International Conference on Computer Vision}
  \acro{ICRA}{International Conference on Robotics and Automation}
  \acro{CVIU}{Computer Vision and Image Understanding}
  \acro{CAD}{Computer-Aided Design}
  \acro{CYB}{Transactions on Cybernetics}
  \acro{THMS}{Transactions on Human-Machine Systems}
  \acro{TPAMI}{Transactions on Pattern Analysis and Machine Intelligence}
  \acro{IJCV}{International Journal of Computer Vision}
\end{acronym}

\end{document}

%%%%%%%%%%%%%%%%%%%%%%%%%% End CV Document %%%%%%%%%%%%%%%%%%%%%%%%%%%%%
